\documentclass{article}
\usepackage[utf8]{inputenc}
\usepackage{multicol}
\usepackage{listings}
\usepackage{verbatim}
\usepackage{color}
\usepackage{geometry}
\usepackage{float}
\usepackage{amsmath}

\usepackage{pdflscape}
\usepackage{hyperref}
\setlength{\belowcaptionskip}{-10pt}
\setlength{\abovecaptionskip}{-30pt}
\floatstyle{boxed} 
\restylefloat{figure}
\usepackage{graphicx}
\definecolor{codegreen}{rgb}{0,0.6,0}
\definecolor{codegray}{rgb}{0.5,0.5,0.5}
\definecolor{codepurple}{rgb}{0.58,0,0.82}
\definecolor{backcolour}{rgb}{0.95,0.95,0.92}

\lstdefinestyle{mystyle}{
	backgroundcolor=\color{backcolour},   
	commentstyle=\color{codegreen},
	keywordstyle=\color{blue},
	numberstyle=\tiny\color{codegray},
	stringstyle=\color{codepurple},
	basicstyle=\footnotesize,
	breakatwhitespace=false,         
	breaklines=true,                 
	captionpos=b,                    
	keepspaces=true,                 
	numbers=left,                    
	numbersep=5pt,                  
	showspaces=false,                
	showstringspaces=false,
	showtabs=false,                  
	tabsize=2
}

\lstset{style=mystyle}
\title{Image Processing Project Proposal\\
Digital face makeup generation}

\author{\textbf{Authors:}Aqeel Labash, Vadym Ponomarov\\ 
	\textbf{Supervisor} Gholamreza Anbarjafari}
\date{23 March 2016}

\geometry{
	a4paper,
	total={170mm,257mm},
	left=10mm,
	top=5mm,
}
\begin{document}
	\maketitle
\section*{Abstract}
In this project our aim is to implement an automated algorithm on face makeup building. It can be done by processing the face and recognizing face parts as accurate as possible and adding the required modifications on colors/hue that correspond to face parts.
\section*{Researching required information/techniques}
\begin{enumerate}
	\item Research basic Pattern Recognition techniques.
	\item Study some papers about the topic, specifically:
	\begin{enumerate}
		\item \href{http://users.cecs.anu.edu.au/~adhall/DhallSharmaBhattKhanISVC2009.pdf}{Adaptive Digital Makeup}
		\item \href{https://www.comp.nus.edu.sg/~tsim/documents/face_makeup_cvpr09_lowres.pdf}{Digital Face Makeup by Example}
		\item \href{http://web.stanford.edu/class/ee368/Project_Autumn_1516/Reports/Wut.pdf}{Digital Makeup Face Generation}
	\end{enumerate} 
\end{enumerate}
\section*{An approximate digital makeup algorithm}
\begin{enumerate}
	\item Image resize for better fitting.
	\item Perform skin segmentation, face detection (probably some ideal high pass filter resulting in a binary image). Possibly, extract several face layers, like structure/color/details.
	\item Detect the face parts (eyes, lips, etc) within the detected face.
	\item Maybe gender determination (optional).
	\item	Skin tone preprocessing.
	\item	Getting some database of makeups/faces would be nice.
	\item	Apply digital makeup:
	\begin{enumerate}
			\item	apply some smoothing filter to remove unwanted details.
			\item	perform color balancing / image enhancement.
			\item apply makeup on face regions (lips, eyes, cheeks etc.) based on reference database or user preference. Possibly, use a reference wanted-makeup image of a different person. Use some blending methods.
	\end{enumerate}
\end{enumerate}    
\section*{General Info}
\textbf{Programming Language:}Matlab or python (only one of them not both)
\end{document}